	\section {Системи керування вмістом}

\Authors{Галиняк Віктор Миколайович}
\aff{Балинський ліцей Смотрицької селещної ради Хмельницької області}
\textbf{Систе́ма керува́ння вмі́стом} (СКВ; англ. Content Management System, CMS) — програмне забезпечення для організації вебсайтів чи інших інформаційних ресурсів в Інтернеті чи окремих комп'ютерних мережах.
Існують сотні, а може, навіть й тисячі доступних CMS — систем. Завдяки їх функціональності ці системи можна використовувати в різних компаніях. Незважаючи на широкий вибір інструментальних та технічних засобів, наявних в CMS, існують загальні для більшості типів систем характеристики.
Перші СКВ були розроблені у великих корпораціях для організації роботи з документацією. У 1995-му від компанії CNET відокремилася окрема компанія Vignette, яка започаткувала ринок для комерційних СКВ. З часом діапазон продукції розширювався і дедалі більше інтегрувався у сучасні мережеві рішення аж до популярних вебпорталів.
Багато сучасних СКВ поширюються як безкоштовні і легкі у встановленні (інсталяції) програми, які розробляються під ліцензією GNU/GPL групами ентузіастів.
Системи управління вебсайтом часто розраховані на роботу у певному програмному середовищі. Наприклад, система MediaWiki, під управлінням якої працює Вікіпедія, написана мовою програмування PHP і зберігає вміст і налаштування у базі даних типу MySQL або PostgreSQL; тому для її роботи потрібно, щоб на сервері, де вона розміщена, були встановлені вебсервер (Apache, IIS чи інший), підтримка PHP та системи керування базами даних MySQL або PostgreSQL, а також, в разі необхідності, додаткові програми для обробки зображень чи математичних формул. Такі вимоги є досить типовими для відкритих СКВ.
\subsection {Різновиди СКВ}
\textbf{Web content management systems} для управління вебсайтами (наприклад, енциклопедіями, подібними до Вікіпедії, онлайн-виданнями, блогами, форумами, корпоративними чи персональними вебсторінками та ін.)
\textbf{Транзакційні СКВ} для забезпечення транзакцій у електронній комерції.
\textbf{Інтегровані СКВ} для роботи з документацією на підприємствах.
\textbf{Електронні бібліотеки} (Digital Asset Management) для забезпечення циклу життя файлів електронних медіа (відео, графічн., презентації тощо).
\textbf{Системи для забезпечення циклу життя документації} (інструкції, довідники, описи).
\textbf{Освітні СКВ} — системи для організації Інтернет курсів та відповідного циклу життя документації. Наприклад:
Системи, що мають українську локалізацію:

\subsection{Joomla}

\textbf{CMS Joomla} — (вимова: \textbf{«Джу́мла»}) — відкрита універсальна система керування вмістом для публікації інформації в інтернеті. Підходить для створення маленьких і великих корпоративних сайтів, інтернет порталів, онлайн-магазинів, сайтів спільнот і персональних сторінок. З особливостей Joomla можна відзначити: гнучкі інструменти управління обліковими записами, інтерфейс для управління медіа-файлами, підтримка створення багатомовних варіантів сторінок, система управління рекламними кампаніями, адресна книга користувачів, голосування, вбудований пошук, функції категоризації посилань і обліку кліків, \textbf{WYSIWYG}-редактор, система шаблонів, підтримка меню, управління новинними потоками, XML-RPC API для інтеграції з іншими системами, підтримка кешування сторінок і великий набір готових доповнень.
\textbf{Joomla!} написана на мові PHP з використанням архітектури MVC. Для збереження інформації використовується база даних MySQL, PostgreSQL чи MS SQL.
\textbf{Joomla!} — вільне програмне забезпечення, захищене ліцензією GPL.
\begin{figure}[h]
	\centering
	\includegraphics[width=0.5\linewidth]{./images-viktor/joomla.png}
	\caption{
		\centering
		Логотип Joomla.}
\end{figure}

\subsection{Drupal}

\textbf{Drupal}  — популярна вільна модульна система керування вмістом (СКВ, англ. CMS) з відкритим вихідним кодом, написана на мові програмування PHP та розповсюджується за ліцензією GNU.
\textbf{Drupal} використовують як back end фреймворк для різних вебсайтів, від особистих блогів до корпоративних та державних сайтів.[9] Drupal також використовується у системах управління знаннями та для ділової співпраці.
\textbf{Drupal} може працювати у таких популярних системах як Windows, Mac OS X, Linux, власне, на будь-якій платформі, яка підтримує роботу вебсервера Apache, Nginx, Lighttpd або Microsoft IIS; також потрібна наявність системи керування базами даних MySQL/MariaDB, PostgreSQL 8.3, SQLite чи інші комерційні. 
Активними борцями за нове життя були \textbf{копищанські жінки}. Так, у постійно діючих комісіях при сільраді їх працювало 4, у комнезамі — 3, у касі взаємодопомоги — 3, у шкільній раді — 3, у правлінні кредитного товариства — 3. На з’їзд жінок Олевського району 1925 року від Конища поїхало 16 жінок.
\begin{figure}[h]
	\centering
	\includegraphics[width=0.5\linewidth]{./images-viktor/drupal.png}
	\caption{
		\centering
		Логотип Drupal.}
\end{figure}

\subsection{Moodle}

\textbf{Moodle} (акронім від Modular Object-Oriented Dynamic Learning Environment — модульне об'єктно-орієнтоване динамічне навчальне середовище) — навчальна платформа, призначена для об'єднання педагогів, адміністраторів і учнів (студентів) в одну надійну, безпечну та інтегровану систему для створення персоналізованого навчального середовища.
Moodle — це безкоштовна, відкрита (Open Source) система управління навчанням. Вона реалізує філософію «педагогіки соціального конструктивізму» та орієнтована насамперед на організацію взаємодії між викладачем та учнями, хоча підходить і для організації традиційних дистанційних курсів, а також підтримки очного навчання.
Moodle перекладена на десятки мов, в тому числі й українською. Система використовувалась у 2014 році — у 197, у 2019 р —229 країнах світу, понад 90 тисяч офіційно зареєстрованих сайтів що працюють на Moodle.
\begin{figure}[h]
	\centering
	\includegraphics[width=0.5\linewidth]{./images-viktor/moodle.png}
	\caption{
		\centering
		Логотип Drupal.}
\end{figure}





